\documentclass[12pt, centerh1]{article}

\textwidth=165mm \headheight=0mm \headsep=10mm \topmargin=0mm
\textheight=220mm %\footskip=1.5cm
\oddsidemargin=0mm

\RequirePackage[colorlinks,citecolor=blue,urlcolor=blue]{hyperref}
\usepackage{amsmath, amssymb,natbib}

\usepackage{subcaption}
\usepackage{graphicx,bm}
\usepackage{color}
 \usepackage[table]{xcolor}
\usepackage{longtable}
\usepackage{amsthm}
\usepackage[mathscr]{euscript}
\usepackage{relsize}
\newcolumntype{P}[1]{>{\centering\arraybackslash}p{#1}}
\usepackage{rotating}
\usepackage{eurosym}
\usepackage{colonequals}
\usepackage{bbm}
\usepackage{pbox}
\usepackage{booktabs}
\usepackage{dsfont}
\usepackage{authblk}
\usepackage{lscape}
\usepackage{physics}
\usepackage{siunitx}
\usepackage{svg}
\usepackage{float}
\usepackage{multirow}
\usepackage{nomencl}
\makenomenclature

% % Uncomment to Exclude All Figures
% % Comment to Include All Figures
% \usepackage{comment}
% \excludecomment{figure}
% \let\endfigure\relax



% Set default image directory
\graphicspath{ {../imgs/} }  


\renewcommand*\abstractname{Abstract}

\title{An Analytic and Numerical Study on the Goodwin and Goodwin-Keen Economics Models} % 
% Old title: The Evolution of the Goodwin Economics Model
\author{\qquad Romi Lifshitz$^{1}$ \qquad\  Arthur M\'endez-Rosales$^{2}$ \qquad\  Sara Saad$^{3}$ \\ Grant Forsythe$^{4}$ \qquad Gheeda Mourtada$^{4}$ \qquad Jacob Ronen Keffer$^{5}$}

\date{
{\footnotesize $^1$ Department of Arts and Science, McMaster University, ON, Canada\\[-6pt]
$^2$ Department of Engineering Physics, McMaster University, ON, Canada \\[-6pt]
$^3$ Department of Electrical and Computer Engineering, McMaster University, ON, Canada\\[-6pt]
$^4$ Department of Mathematics and Statistics, McMaster University, ON, Canada\\[-6pt]
$^5$Department of Chemistry and Chemical Biology, McMaster University, ON, Canada\\[-6pt]}
}

% \author[1]{Romi Lifshitz}
% \author[2]{Arthur M\'endez-Rosale}
% \author[3]{Sara Saad}
% \author[4]{Grant Forsythe}
% \author[4]{Gheeda Mourtada}
% \author[5]{Jacob Ronen Keffer}

% \affil[1]{\begin{small}Department of Arts and Science, McMaster University, ON, Canada
% \end{small}}
% \affil[2]{\begin{small}Department of Engineering Physics, McMaster University, ON, Canada\end{small}}
% \affil[3]{\begin{small}Department of Electrical and Computer Engineering, McMaster University, ON, Canada
% \end{small}}
% \affil[4]{\begin{small}Department of Mathematics and Statistics, McMaster University, ON, Canada \end{small}}
% \affil[5]{\begin{small}Department of Chemistry and Chemical Biology, McMaster University, ON, Canada \end{small}}

%%%%%%%%%%%%%%%
\linespread{1.5}
\pdfminorversion=4

\begin{document}

\maketitle
\vspace{-8mm} % Push abstract up by 8mm
\begin{abstract}
% Example abstract. Source: https://arxiv.org/pdf/2011.14522.pdf
In every economy there exist many factors impacting long-term equilibria, some of which have been accounted for by various models. Here, we examine the differences between the Goodwin and Goodwin-Keen models which seek to explain economic dynamics. To do so, we present long-term equilibria and conduct a brief sensitivity study. The results illustrate that the Goodwin model presents only one realistic equilibrium, that the Goodwin-Keen model's non-trivial solution is an extreme economic scenario, and that the long-term economic equilibria are impacted by the initial state variable conditions. These findings motivate the pursuit of a deeper understanding of economic dynamics, as it paves the way for better predictions of economic events. \\ \textbf{Keywords}: Goodwin model, Keen model, Endogenous economics, Prey-predator dynamics, Economic instability, Simulation Study

\end{abstract}
\newpage

\section{Introduction}
% At least one page, at most page and a half.
% Goodwin Model Paragraph
\noindent Exogenous economics models are those that assume an economy is stable and operates in equilibrium, such that only external factors can cause a potential crash \citep{ganti_2019}. However, the real world reflects that a macro-economy can itself destabilize due to internal factors. These factors are wage share, employment rate, and private debt, compounded with the systems placed by governments to ensure stability \citep{minsky1992financial}. In 1967, R. M. Goodwin proposed an endogenous economics model that, at its core, mimics Lotka-Volterra prey-predator dynamics \citep{goodwin1982growth}. Although issues with this model have been observed since its conception \citep{harvie2000testing}, many still agree that the model holds value from a qualitative perspective due to its attention to the interplay and dynamic behaviour of internal economic factors \citep{flaschel2016mathematical}. 

% Minsky Hypothesis and the Goodwin-Keen Model Paragraph (TBD)
In 1992, Hyman Minsky hypothesized that economies can destabilize due to internal factors including those considered by Goodwin \citep{minsky1992financial}. Minsky's hypothesis, known as the \emph{Instability Hypothesis}, states that an economy can become fragile because capitalists are willing to incur credit-based debt in pursuit of capital gains. Minsky also identified some limitations of the Goodwin model such as its inability to account for long-term and short-term debts separately, and the impact that income inequality and government regulations have on economic instability \citep{minsky1992financial}. However, Minksy did not represent the hypothesis mathematically. Realizing its importance, Steve Keen \citep{keen1995finance} introduced its key ideas into the Goodwin model––resulting in the now-known Goodwin-Keen Model. 

% Purpose Paragraph
In isolation, these models exhibit sensible and closed-form solutions as presented by \citet{goodwin1982growth} and \citet{grasselli2012analysis}. It is of particular interest to understand the impact of Keen's addendum of Minsky's hypothesis to the predictive value of the model since, despite their shortcomings \citep{harvie2000testing, moura2013testing}, there is value in understanding the underlying prey-predator dynamics of the economy upon which the models are based. As such, in this report, we examine Keen's introduction of private debt into Goodwin's original model.\\

% Closing Paragraph
Specifically, we take an in-depth look at the Goodwin model with respect to its long-term behaviour and stability. This is demonstrated both analytically, and corroborated through numerical methods. An equivalent analysis is performed on the Goodwin-Keen model. The results for each model are compared, focusing on their relative long-term predictive value. Finally, conclusive remarks are made based on the value of these models as potential, real-world, predictive tools.

% \newpage % DONT FORGET TO REMOVE THIS. 

\section{Methodology}
% This is where you write your mathematical definitions, or dataset acquisitions, and or cleaning of data. 
% Arthur's Collab. and Synthesized Code: https://colab.research.google.com/drive/1LJISFLnZGBlSe2oGJ1wf6I0rD-wq9Nes?usp=sharing

Python \citep{rossum1995python} and various scientific packages were used to construct the models. All model parameters are in accordance with the assumptions used by \citet{grasselli2012analysis}, as shown in Table \ref{tab:parameters} below. The entire project code can be viewed on \href{https://github.com/grantwforsythe/math3mb3}{GitHub}.
\begin{table}[!h]
\caption{Model Parameters used by \citet{grasselli2012analysis}.}
\label{tab:parameters}
\centering
\begin{tabular}{|c|c|l|}
\hline
\textbf{Parameter }     & \textbf{Value}            & \textbf{Definition}                       \\
\hline
$\alpha$                & $0.025$                   & Technological Growth Rate                 \\
$\beta$                 & $0.02$                    & Population Growth Rate                    \\
$\delta$                & $0.01$                    & Depreciation Rate                 \\ \hline
$\Phi_0$                & $0.04/(1-0.04^2)$         & \multirow{2}{*}{Phillips Curve Parameters}\\
$\Phi_1$                & $0.04^3/(1-0.04^2)$       & ~                                 \\ \hline
$\kappa_0$             & $-0.0065$                 & \multirow{3}{*}{Rate of Investment Parameters}   \\
$\kappa_1$              & $\mathrm e^{-5}$          & ~                                         \\
$\kappa_2$              & $20$                      & ~                                 \\ \hline
$r$                     & $0.03$                    & Real Interest Rate                        \\
$\nu$                   & $3$                       & Capital-to-output Ratio                   \\
\hline
\end{tabular}
\end{table}
\subsection{Goodwin Model}
The Goodwin model is described by
\begin{equation}\label{eq:goodwin} 
\begin{split}
    \dot{\lambda} &= \lambda \cdot \left( \frac{1-\omega}{\nu} - \alpha - \beta - \delta \right), \\
    \dot{\omega} &= \omega \cdot (\Phi(\lambda) - \alpha).
\end{split}
\end{equation}
In this model, $\lambda$ represents the fraction of employed workers, and $\omega$ represents the worker's wage share. The Jacobian matrix for the Goodwin system of equations presented as \eqref{eq:goodwin} is defined by
\begin{equation} \label{eq:goodwinJ}
\mathbf J =
\begin{bmatrix}
    \pdv{\lambda}\dot{\lambda} & \pdv{\omega}\dot{\lambda}\\[1ex]
    \pdv{\lambda}\dot{\omega} & \pdv{\omega}\dot{\omega}
\end{bmatrix}.
\end{equation}
This Jacobian matrix was symbolically determined using various methods (i.e. \texttt{solvers.solve}, \texttt{symbols}, \texttt{Matrix}) from the SymPy library \citep{SymPy}.  The equilibrium points and the corresponding eigenvalues of $\mathbf J$ were also symbolically determined using these methods. To corroborate these results, the parameters in Table \ref{tab:parameters} were used to numerically solve the system. This was done using the \texttt{integrate.odeint} method from the SciPy library \citep{2020SciPy-NMeth}. Initial conditions to the system were selected based on \citet{grasselli2012analysis}. The Matplotlib \citep{matplotlib} graphic library was used to generate representative plots of the resulting numerical solution.

\subsection{Goodwin-Keen Model}
We consider the Goodwin-Keen model as the system of differential equations defined by 
\begin{equation} \label{eq:keen}
\begin{split}
    \dot{\lambda} &= \lambda \cdot \left( \frac{\kappa(\pi)}{\nu} - \alpha - \beta - \delta \right),\\
    \dot{\omega} &= \omega \cdot (\Phi(\lambda) - \alpha),\\
    \dot{d} &= d\cdot\left(r-\frac{\kappa(\pi)}{\nu}+\delta\right)+\kappa(\pi)-(1-\omega), \\
    \pi &= 1-\omega-rd.
\end{split}
\end{equation}
Where the newly introduced variable $d$ represents the debt ratio of the economy. Additionally, a nonlinear increasing function $\kappa(\pi)$ now represents a non-linear rate of new investment in the economy \citep{grasselli2012analysis}, and $\pi$ represents the net profits share\footnote{Note $\pi$ is an implicit part of the model and does not represent a state variables. It represents the net profit made from capital investments \citep{grasselli2012analysis}, and also serves to simplify the analysis.}. The Jacobian matrix for the system is now defined by 
\begin{equation} \label{eq:keenJ}
\mathbf J =
\begin{bmatrix}
    \pdv{\lambda}\dot{\lambda} & \pdv{\omega}\dot{\lambda} & \pdv{d}\dot{\lambda}\\[1ex]
    \pdv{\lambda}\dot{\omega} & \pdv{\omega}\dot{\omega} & \pdv{d}\dot{\omega}\\[1ex]
    \pdv{\lambda}\dot{d} & \pdv{\omega}\dot{d} & \pdv{d}\dot{d}
\end{bmatrix}.
\end{equation}
Given the increased complexity of this model, equilibrium points were determined numerically for a variety of initial conditions using the \texttt{integrate.odeint} method from the SciPy library \citep{2020SciPy-NMeth} and simulating for 1000 steps. The closed-form Jacobian matrix was determined using the SymPy library \citep{SymPy}, and its eigenvalues were then determined using the SymPy \texttt{eigenvects} method. A \emph{basin of attraction} study was then performed on this model using the parameters in Table \ref{tab:parameters}. Long-term model convergence was asserted by verifying that the real part of the eigenvalues of \eqref{eq:keenJ}, for various initial conditions, had a value less than zero. All results were plotted using the Matplotlib \citep{matplotlib} graphic library.

\section{Application}
% You need to talk about the results of your model. 
% Arthur's Collab. and Synthesized Code: https://colab.research.google.com/drive/1LJISFLnZGBlSe2oGJ1wf6I0rD-wq9Nes?usp=sharing

\subsection{The Goodwin Model}
The following is a detailed study on the system behaviour demonstrated by the Goodwin model. We begin by noting that in \eqref{eq:goodwin} the Philips curve, $\Phi(\lambda)$, has not been explicitly defined. When the model was proposed \citep{goodwin1982growth}, $\Phi(\lambda)$ was assumed to be a linear function of employment rate. This would mean that a higher fraction of employment leads to a rising wage share. Nevertheless, \citet{goodwin1982growth} admitted that this was an empirical and disputable assumption. On that account, \citet{keen1995finance} proposed \eqref{eq:keenPhil} as a more realistic model. The introduced function models a worker population more willing to accept wage cuts at higher levels of unemployment (i.e. lower $\lambda$). Conversely, as employment rates increase, workers will demand real wages that asymptotically increase (i.e. $\lambda\to1$). This function was also adopted by \citet{grasselli2012analysis} and incorporated into the Goodwin model
\begin{equation} \label{eq:keenPhil}
    \Phi(\lambda) = \frac{\Phi_1}{(1-\lambda)^2}-\Phi_0.
\end{equation}
Thus, from \eqref{eq:goodwinJ}, it follows that
\begin{equation} \label{eq:jac_good}
    \mathbf J = \begin{bmatrix}
        \frac{1-\omega}{\nu}-\alpha-\beta-\delta & -\frac{\lambda}{\nu}\\[1ex]
        \frac{2\Phi_1\omega}{(1-\lambda)^3} & \frac{\Phi_1}{(1-\lambda)^2}-\Phi_0-\alpha
\end{bmatrix}.
\end{equation}
The eigenvalues of \eqref{eq:jac_good} determine the nature of the stability of the model at equilibrium. As it turns out, the Goodwin model has two equilibrium points besides the trivial solution $(\lambda^\ast,\omega^\ast)=(0, 0)$. These include
\begin{equation} \label{eq:goodwin_eqm}
\begin{split}
    (\lambda_\pm^\ast, \omega^\ast) &= \left(1\pm\sqrt{\frac{\Phi_1}{\alpha + \Phi_0}},\quad  1-\nu\cdot(\alpha + \beta + \delta)\right)
\end{split}.
\end{equation}
We first must note that all the parameters in \eqref{eq:goodwin_eqm}, as described in Table \ref{tab:parameters}, are positive such that $\sqrt{\frac{\Phi_1}{\alpha + \Phi_0}} \in\mathbb R > 0$. Moreover, $\lambda$ is restricted to the bounded interval $[0, 1]$. This means the only economically feasible equilibrium point is $(\lambda_-^\ast, \omega^\ast)$, since $\lambda_+^\ast$ is inevitably greater than unity. As $(\lambda_-^\ast, \omega^\ast)$ is the only meaningful equilibrium point, we simply denote it as $(\lambda^\ast, \omega^\ast)$. The valid equilibrium point can be incorporated into \eqref{eq:jac_good} and yield the following eigenvalues
\begin{equation} \label{eq:goodwin_eig}
    \pm\sqrt 2\cdot\sqrt{\xi_1 - \xi_2},
\end{equation}
where
\begin{equation*}
\begin{split}
    \xi_1 &= \frac{\Phi_0 + \alpha}{\nu} + \frac{\sqrt{\Phi_1\cdot(\Phi_0 + \alpha)}(\Phi_0+\alpha)(\alpha+\beta+\delta)}{\Phi_1},\\
    \xi_2 &= \frac{\sqrt{\Phi_1\cdot(\Phi_0 + \alpha)}(\Phi_0 + \alpha)}{\Phi_1\nu} + (\Phi_0+\alpha)(\alpha+\beta+\delta).
    \end{split}
\end{equation*}
These expressions hold an integral detail about the nature of the equilibrium points. Namely, whether the solution is cyclic in nature. Clearly, if $\xi_1>\xi_2$ then the eigenvalues are solely real and must necessarily diverge (since one of the eigenvalues is strictly positive). However, if $\xi_1<\xi_2$ we can guarantee that both eigenvalues will yield a converging solution, and said solution will be cyclical in nature. Importantly, scenario one, while mathematically plausible, is also economically unrealistic because it suggests that either $\lambda$ or $\omega$ must exceed unity. To proceed, the parameter values defined in Table \ref{tab:parameters} are introduced. Evaluating expression \eqref{eq:goodwin_eig} yields the (approximate) imaginary eigenvalues $\pm0.747558\sqrt{2}i$.
With these parameters we can also determine the equilibrium point to be approximately
\begin{equation}
    (\lambda^\ast, \omega^\ast) = (0.968612, 0.835000).
\end{equation}
\noindent
To verify the aforementioned results, the system was numerically solved. Figure \ref{fig:goodwin} shows the behaviour of the model for 100 steps (equivalent to 100 years). 
\begin{figure}[H]
    \centering
    \includesvg[scale=0.7]{goodwin_model.svg}
    \caption{Goodwin Model Limit Cycle, Simulated for 100 Steps.}
    \label{fig:goodwin}
\end{figure}
\noindent With the selected parameters, the model must be convergent and exhibit a cyclical behaviour. This behaviour is clearly demonstrated by the phase portrait depicted in Figure \ref{fig:goodwin_phase}. Here, the model was simulated for 1000 steps, and it is evident that the system, while not static, had reached a form of equilibrium--particularly a limit cycle. This plot also shows the value of the equilibrium point $(\lambda^\ast, \omega^\ast)$ relative to the equilibrium orbit of the system.
\begin{figure}[H]
    \centering
    \includesvg[scale=0.7]{goodwin_phase.svg}
    \caption{Phase Portrait of the Goodwin Model Limit Cycle, Simulated for 100 Steps.}
    \label{fig:goodwin_phase}
\end{figure}

\subsection{The Goodwin-Keen Model}
The Jacobian matrix for the Goodwin-Keen model makes use of both \eqref{eq:keenPhil} and 
\begin{equation*} \label{eq:keenKappa}
    \kappa=\kappa(\pi) = \kappa_0 + \kappa_1\mathrm e^{\kappa_2\pi}.
\end{equation*}
Before we present the matrix, we note the use of the notational simplification
\begin{equation*}
    \kappa^\prime = -\pdv{\kappa}{\omega} = -\frac{1}{r}\pdv{\kappa}{d} = \kappa_1\kappa_2\mathrm e^{\kappa_2\pi}.
\end{equation*}
Thus the Jacobian matrix for the Goodwin-Keen model is given by
\begin{equation} \label{eq:jac_keen}
    \mathbf J = %
    \begin{bmatrix}
        \frac{\kappa-\nu(\alpha+\beta+\delta)}{\nu} & -\frac{\lambda\kappa^\prime}{\nu} & -\frac{\lambda r\kappa^\prime}{\nu} \\[0.5em]
        \frac{2\Phi_1\omega}{(1-\lambda)^3} & \frac{\Phi_1}{(1-\lambda)^2}-\Phi_0-\alpha & 0\\[0.5em]
        0 & \frac{(d-\nu)\kappa^\prime+\nu}{\nu} & \frac{r\cdot(d-\nu)\kappa^\prime}{\nu}+ \delta + r
    \end{bmatrix}.
\end{equation}
The numerical solution provided values for the equilibrium points. One illustrative example of the solution of the model is shown in Figures \ref{fig:keen} and \ref{fig:keen_phase} using the parameter values in Table \ref{tab:parameters} and initial conditions used by \citet{grasselli2012analysis}.
\begin{figure}[H]
    \centering
    \includesvg[scale=0.7]{keen_model.svg}
    \caption{Goodwin-Keen Model Convergence to Equilibrium Points, Simulated for 300 Steps.}
    \label{fig:keen}
\end{figure}

\begin{figure}[H]
    \centering
    \includesvg[scale=0.7]{keen_phase.svg}
    \caption{Goodwin-Keen Equilibrium Three-dimensional Phase Diagram.}
    \label{fig:keen_phase}
\end{figure}
\noindent The results of the simulation study for the basin of attraction of this model are consolidated in Figure \ref{fig:keen_study}.
\begin{figure}[H]
    \centering
    \includesvg[scale=0.7]{keen_study.svg}
    \caption{Goodwin-Keen Basin of Attraction.}
    \label{fig:keen_study}
\end{figure}
\noindent This figure shows that, of the total 1000 simulations performed, only 102 sets of initial conditions converged, and the remaining 898 did not. Moreover, all converging events lead to the same equilibrium point
\begin{equation}
    (\lambda^\ast, \omega^\ast, d^\ast) = (0.968612, 0.86053, 0.070191).
\end{equation}
The simulations that lead to a diverging event yielded floating-point values where $\lambda$ and $\omega$ were consistently on the order of \num{e-20} (or lower) and $d$ on the order of \num{e19} (or higher). This suggests that another ``equilibrium point'' can be found at
\begin{equation}
    (\lambda^\times, \omega^\times, d^\times) = (0, 0, +\infty).
\end{equation}

\subsection{Model Comparison} 
It is known that the Goodwin model suffers from many limitations \citep{harvie2000testing, moura2013testing}. Nevertheless, the model has persisted. The Goodwin model supplies economists with insight into the prey-predator relationships between capitalists and workers \citep{goodwin1982growth}. We have concluded from our study that this model only has one economically realistic equilibrium. More importantly, this equilibrium can only be divergent (but economically unfeasible) or oscillatory. This feature has been heavily criticized since, in reality, economies are liable to failures. The Goodwin model does not consider such a scenario, as it does not describe the economy in sufficient detail. However, what it does offer is a concise formulation and a stepping-stone for more complex and accurate models.

Such is the case for the Goodwin-Keen model. What Steve Keen accomplished was the persistence of the prey-predator nature of the original model with the possibility of exhibiting an economic failure scenario \citep{keen1995finance}. Here, $(\lambda^\times, \omega^\times, d^\times)$ represents this critical contribution to the overall model. Unlike with the Goodwin model, where the trivial solution holds no economic value, the divergent solution for the Goodwin-Keen model describes the real scenario of an economy with maximum debt, absolute unemployment, and no wage share for the working population \citep{minsky1992financial}. Our simulation study also shows that \emph{ceteris paribus}, the nature of the long-term equilibrium of the economy, is dependant on the initial conditions of the state variables $\lambda$, $\omega$, and $d$. That is, economies described by the same parameters can fail or stabilize solely as a result of where they begin.

% If your model or problem set has parameters your change, this is where you would write about it. 
% Minimum 1 and a half pages. 

\section{Conclusion}
Throughout this report, the Goodwin and Goodwin-Keen models have been described qualitatively, and presented mathematically. The behaviours of the economic models have been explored numerically and analytically. The tendency of the Goodwin model to converge to an oscillatory equilibrium state or diverge has been demonstrated. The Goodwin-Keen model, which can show more realistic features representative an economic collapse, has been investigated, and simulations have been performed. A thorough comparison of the models has interpreted their behaviours. This study has concluded that the Goodwin model produces unrealistic long-term behaviour that cannot be used to accurately describe an economy. It has, however, led to many more economically representative models based on the original prey-predator dynamic, such as the Goodwin-Keen model. As demonstrated in this report, Keen's improved model can reasonably simulate a economic crash (instability). The ability to simulate and predict an incoming economic crash, such as a recession or depression, gives governments invaluable foresight. This may allow policy-makers to enact regulations for avoiding, or at the very least, curbing the severity of the crash. Our findings motivate the pursuit of a deeper understanding of economic dynamics, as it paves the way for better predictions of economic events.
% Housing Prices: Future projections show this, from the model. Then write about what the future results indicate. 
% SIR Model: If you decrease the infection rate by 20\%, you will get a peak decrease of 15\%. 
% Sport Statistics: This model indicates player outcomes for this and this team to win these type of games. 
% This is half a page long.

\section{Acknowledgments}
We would like to thank Nik Počuča (McMaster University) for encouraging us to think about and explore this problem. We would also like to thank Daniel Presta for useful discussions and invaluable feedback throughout the project.

\bibliographystyle{chicago}
\bibliography{bibliography.bib}

\newpage
\mbox{}
\nomenclature[1]{$\lambda$}{Employment rate}
\nomenclature[2]{$\omega$}{Wage share}
\nomenclature[3]{$d$}{Debt ratio}

% \nomenclature{$L$}{Total labour employed}
% \nomenclature{$N$}{Total labour force}
% \nomenclature{$w$}{Real wages per unit of labour}
% \nomenclature{$a$}{Labour productivity}
% \nomenclature{$D$}{Amount of debt in real terms}
% \nomenclature{$Y$}{Total yearly Output}
% \nomenclature{$K$}{Capital stock}

\nomenclature{$\nu$}{Constant capital-to-output ratio}
\nomenclature{$\kappa$}{Acceleration relation for the total real capital stock}
\nomenclature{$\beta$}{Population growth rate}
\nomenclature{$\delta$}{Constant depreciation rate}
\nomenclature{$\alpha$}{The exponential growth of disembodied productivity growth rate, Technological growth}
\nomenclature{$r$}{Constant real interest rate}
\nomenclature{$\Phi(\lambda)$}{Function of the rate of employment, Phillips curve}
\printnomenclature

% \newpage
% \section*{Appendix}

\end{document}
