\documentclass[12pt, centerh1]{article}

\textwidth=165mm \headheight=0mm \headsep=10mm \topmargin=0mm
\textheight=220mm %\footskip=1.5cm
\oddsidemargin=0mm

\RequirePackage[colorlinks,citecolor=blue,urlcolor=blue]{hyperref}
\usepackage{amsmath, amssymb,natbib}
%\usepackage[mathscr]{euscript}
%\usepackage{mathrsfs}
\usepackage{subcaption}
\usepackage{graphicx,bm}
\usepackage{color}
\usepackage{subcaption}
 \usepackage[table]{xcolor}
\usepackage{longtable}
\usepackage{amsthm}
\usepackage[mathscr]{euscript}
\usepackage{relsize}
\newcolumntype{P}[1]{>{\centering\arraybackslash}p{#1}}
\usepackage{rotating}
\usepackage{eurosym}
\usepackage{colonequals}
\usepackage{bbm}
\usepackage{pbox}
\usepackage{booktabs}
\usepackage{dsfont}
\usepackage{authblk}
\usepackage{lscape}
\usepackage{physics}
%\usepackage{floatrow}

\usepackage{nomencl}
\makenomenclature


\renewcommand*\abstractname{Abstract}

\title{The Evolution of the Goodwin Economics Model} % 


\author[1]{Romi Lifshitz}
\author[2]{Arthur M\'endez-Rosale}
\author[3]{Sara Saad}
\author[4]{Grant Forsythe}
\author[4]{Gheeda Mourtada}
\author[5]{Jacob Ronen Keffer}

\affil[1]{\begin{small}Department of Arts and Science, McMaster University, ON, Canada
\end{small}}
\affil[2]{\begin{small}Department of Engineering Physics, McMaster University, ON, Canada\end{small}}
\affil[3]{\begin{small}Department of Electrical and Computer Engineering, McMaster University, ON, Canada
\end{small}}
\affil[4]{\begin{small}Department of Mathematics and Statistics, McMaster University, ON, Canada \end{small}}
\affil[5]{\begin{small}Department of Chemistry and Chemical Biology, McMaster University, ON, Canada \end{small}}

%%%%%%%%%%%%%%%
\linespread{1.5}
\pdfminorversion=4



\begin{document}


\maketitle
\begin{abstract}
% Example abstract. Source: https://arxiv.org/pdf/2011.14522.pdf
As its width tends to infinity, a deep neural network’s behavior under gradientdescent can become simplified and predictable (e.g. given by the Neural TangentKernel (NTK)), if it is parametrized appropriately (e.g. the NTK parametrization). However, we show that the standard and NTK parametrizations of a neural network do not admit infinite-width limits that canlearnfeatures, which is crucial for pre-training and transfer learning such as with BERT. We propose simple modificationsto the standard parametrization to allow for feature learning in the limit.  UsingtheTensor Programstechnique, we derive explicit formulas for such limits. OnWord2Vec and few-shot learning on Omniglot via MAML, two canonical tasksthat rely crucially on feature learning, we compute these limits exactly. We findthat they outperform both NTK baselines and finite-width networks, with the latterapproaching the infinite-width feature learning performance as width increases...

% \noindent\textbf{Keywords}: UK Biobank, accelerometer data, matrix variate, telemetric data, Cox regression, hazard ratios, big data, high-dimensional data, large scale data processing.
\end{abstract}
\newpage % DONT FORGET TO REMOVE THIS. 

\section{Introduction}
% At least one page, at most page and a half.
% Goodwin Model Paragraph
\noindent Exogenous economics models are those that assume the economy is stable and operates in equilibrium, such that only external factors can cause a potential crash \citep{ganti_2019}. However, the real world reflects that a macro-economy can itself destabilize due to internal factors such as wage share, employment rate, and private debt, compounded by the systems put in place by governments to ensure stability \citep{minsky1992financial}. R. M. Goodwin proposed an endogenous economics model in 1967 that, at its core, mimics Lotka-Volterra prey-predator dynamics \citep{goodwin1982growth}. Many issues have been observed with this model since its conceptions \color{red}\textbf{CITE}\color{black} but nevertheless, many still agree that it holds value from a qualitative perspective as it draws attention to the dynamic behaviour of internal economic factors \citep{flaschel2016mathematical}. Goodwin model is described by
\begin{align*}\label{eq:goodwin} 
\centering
    \dot{\lambda} &= \lambda \cdot \left( \frac{\kappa(1-\omega-rd)}{\nu} - \alpha - \beta - \delta \right), \\ 
    \dot{\omega} &= \omega \cdot (\Phi(\lambda) - \alpha).
\end{align*}

% Minsky Hypothesis and the Goodwin-Keen Model Paragraph (TBD)

% Data Fitting Paragraph
\noindent The shortcomings that have been identified for both the original Goodwin model and Keen’s integration of Minsky’s hypothesis make themselves clear when these models are compared against real-world data. In isolation, the models can exhibit logical rational and closed-form solutions as presented by \citet{goodwin1982growth} and \citet{grasselli2012analysis}. However, examples where these same models are compared to real-world data \citep{harvie2000testing, moura2013testing} make it clear that the underlying theory has limitations. It is therefore imperative to understand where these models fail to be representative of real-world phenomena, and where they do exhibit some predictive value. 

% Closing Paragraph
\noindent The following report considered the behaviour of the Goodwin model and Steve Keen’s expansion both from an isolated, analytical perspective, and as a tool to model real-world macroeconomics phenomena. The long term equilibrium for both models will be presented along with a brief parameter sensitivity study. Based on these results, numerical methods will be employed to fit employment rate data from the US from 1990 to 2008.


\newpage % DONT FORGET TO REMOVE THIS. 

\section{Methodology}
% This is where you write your mathematical definitions, or dataset acquistions, and or cleaning of data. 
% At most two pages, at least 1 page.
Python \citep{rossum1995python} and various scientific packages were used to construct the models.

\subsection{Data Processing}

\subsection{Goodwin Model}

\[
\mathbf J =
\begin{bmatrix}
    \frac{d\lambda}{dt}\dot{\lambda} & \frac{d\omega}{dt}\dot{\lambda}\\[1ex]
    \frac{d\lambda}{dt}\dot{\omega} & \frac{d\omega}{dt}\dot{\omega}
\end{bmatrix}
\]

\noindent
It follows that:

\[
\mathbf J =
\begin{bmatrix}
    \frac{1-\omega}{v}-\alpha-\beta-\delta & \frac{-\lambda}{v}\\[1ex]
    \frac{2}{(1-\lambda)^3}\Phi_1\omega & \frac{d\Phi_1}{(1-\lambda)^2}-\Phi_0
\end{bmatrix}
\]


\noindent
Calling the \textit{solve} method within the SymPy libary \citep{SymPy}, it returns the symbolic solution\footnote{The first row is the trivial solution.}:
\[
\begin{bmatrix}
    0.0 & 0.0\\[1ex]
    1.0-\frac{0.2}{\sqrt{624.0\alpha+25.0}} & -\alpha v - \beta v -\delta v +1.0\\[1ex]
    1.0+\frac{0.2}{\sqrt{624.0\alpha+25.0}}& -\alpha v - \beta v -\delta v + 1.0 
\end{bmatrix}
\]
\noindent
Then, using the initial model parameters, the solution is:
\[
\begin{bmatrix}
    0.0 & 0.0\\[1ex]
    0.968611758971283 & 0.834999999999999\\[1ex]
    1.03138824102872 & 0.834999999999999 
\end{bmatrix}
\]
Since it is not possible to have an employment rate greater one, the equilibrium points are:
\begin{align*}
    \omega^* &= 0.9686\\
    \lambda^* &= 0.8349
\end{align*}
The stability of the model at equilibrium was determined by finding the eigen values of the Jacobian. This was done by calling the \textit{eig} method from the \textit{linalg} sub module in NumPy \citep{2020NumPy-Array}, which returns the following complex vector:
\[
\begin{bmatrix}
    0.0125+1.05713232i & 0.0125-1.05713232i
\end{bmatrix}
\]
Since both real parts are greater than zero, the model is Lyapunov stable at equilibrium.

\subsection{Goodwin Keen Model}

\begin{align*} 
\centering
    \dot{\omega} &= \omega \cdot (\Phi(\lambda) - \alpha),\\
    \dot{\lambda} &= \lambda \cdot \left( \frac{\kappa(1-\omega-rd)}{\nu} - \alpha - \beta - \delta \right),\\ 
    \dot{d} &= d\cdot\left(r-\frac{\kappa(1-\omega-rd)}{\nu}+\delta\right)+\kappa(1-\omega-rd)-(1-\omega). 
\end{align*}



\subsection{Plot Construction}
\newpage
\section{Application}
You need to talk about the results of your model. 
\subsection{Simulation Study}
If your model or problem set has parameters your change, this is where you would write about it. 
Minimum 1 and a half pages. 

\section{Conclusion}
Housing Prices: Future projections show this, from the model. Then write about what the future results indicate. 
SIR Model: If you decrease the infection rate by 20\%, you will get a peak decrease of 15\%. 
Sport Statistics: This model indicates player outcomes for this and this team to win these type of games. 
This is half a page long. 

This is an example reference \cite{pocuca2020}, 

\bibliographystyle{chicago}
\bibliography{bibliography.bib}

\newpage
\mbox{}
\nomenclature{$\lambda$}{Employment rate}
\nomenclature{$\omega$}{Wage share}
\nomenclature{$d$}{Debt ratio}
\nomenclature{$L$}{Total labour employed}
\nomenclature{$N$}{Total labour force}
\nomenclature{$w$}{Real wages per unit of labour}
\nomenclature{$a$}{Labour productivity}
\nomenclature{$D$}{Amount of debt in real terms}
\nomenclature{$Y$}{Total yearly Output}
\nomenclature{$K$}{Capital stock}
\nomenclature{$\kappa$}{Fixed acceleration relation for the total real capital stock}
\nomenclature{$\beta$}{The population growth rate}
\nomenclature{$\delta$}{Constant depreciation rate}
\nomenclature{$\alpha$}{The exponential growth of disembodied productivity growth rate, Technological growth}
\nomenclature{$r$}{Constant real interest rate}
\nomenclature{$\Phi(\lambda)$}{Non-linear function of the rate of employment, Phillips curve}
\printnomenclature
\newpage
\section*{Appendix}
\end{document}
